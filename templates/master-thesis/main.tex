%%
% The BIThesis Template for experiment report
%
% Copyright 2020-2022 Yang Yating, BITNP
%
% This work may be distributed and/or modified under the
% conditions of the LaTeX Project Public License, either version 1.3
% of this license or (at your option) any later version.
% The latest version of this license is in
%   http://www.latex-project.org/lppl.txt
% and version 1.3 or later is part of all distributions of LaTeX
% version 2005/12/01 or later.
%
% This work has the LPPL maintenance status `maintained'.
%
% The Current Maintainer of this work is Feng Kaiyu.
%
% Compile with: xelatex -> biber -> xelatex -> xelatex

% 默认单面打印 oneside 、硕士论文模板 master

\documentclass[type=master]{bitundergrad}

\BITUndergraduateThesisSetup{
  cover = {
    date = 2022年6月,
  },
  info = {
    classification = TQ028.1,
    UDC = 540,
    title = 形状记忆聚氨酯的合成及其在织物中的应用,
    TITLE = Synthesisand Application on textile of the Shape Memory Polyurethane,
    name = 张三,
    major = 材料科学与工程,
    dept = 材料学院,
    degree = 工学硕士,
    chairman = 王五教授,
    defense_date = 2009年6月1日,
    mentor = 李四教授,
    authorEn = San Zhang,
    schoolEn = Materials Science and Engineering,
    supervisorEn = Prof. Si Li,
    chairmanEn = Prof. Wang Wu,
    degreeEn = Master of Philosophy,
    majorEn = Materials Science and Engineering, 
    defenseDateEn = {June, 12th, 2022},
    keywords = {形状记忆;聚氨酯;织物;合成;应用\textcolor{blue}{(硕士一般选3~6个单词或专业术语,博士一般选3~8个单词或专业术语,且中英文关键词必须对应。)}},
    keywordsEn = shape memory properties; polyurethane; textile; synthesis; application,
  }
}

\RequirePackage[
  defernumbers=true,
  backend=biber,
  style=gb7714-2015,
  gbalign=gb7714-2015,
  gbnamefmt=lowercase,
  gbpub=false,
  doi=false,
  url=false,
  eprint=false,
  isbn=false,
]{biblatex}

\addbibresource{reference/main.bib} % 添加参考文献
\addbibresource{reference/pub.bib}


\usepackage{graphicx}

\begin{document}

%%%%%%%%%%%%%%%%%%%%%%%%%%%%%%
%% 封面
%%%%%%%%%%%%%%%%%%%%%%%%%%%%%%

% 中文封面内容(关注内容而不是表现形式)
% \classification{TQ028.1}
% \UDC{540}
%
% \title{形状记忆聚氨酯的合成及其在织物中的应用}
% \vtitle{形状记忆聚氨酯{P } {D } {F } {A }\rotatebox[origin=c]{-90}{~\LaTeX{}~} 的合成及其在织物中的应用}
% \author{***}
% \institute{**学院}
% \advisor{**教授}
% \chairman{**教授}
% \degree{工学硕士(博士)}
% \major{*****}
% \school{北京理工大学}
% \defenddate{****年*月}
% %\studentnumber{**********}
%
%
% % 英文封面内容(关注内容而不是表现形式)
% \englishtitle{Synthesis and Application on textile of the Shape\\Memory Polyurethane}
% \englishauthor{***}
% \englishadvisor{Prof. **}
% \englishchairman{Prof. **}
% \englishschool{Beijing Institute of Technology}
% \englishinstitute{****}
% \englishdegree{****}
% \englishmajor{****}
% \englishdate{*,****}

% 封面绘制
\MakeCover

\MakePaperBack

\MakeTitle

\MakeOriginality

%
% % 中文信息
% \makeInfo
%
% % 英文信息
% \makeEnglishInfo
%
% %打印竖排论文题目
% \makeVerticalTitle
%
% % 论文原创性声明和使用授权
% \makeDeclareOriginal


%%%%%%%%%%%%%%%%%%%%%%%%%%%%%%
%% 前置部分
%%%%%%%%%%%%%%%%%%%%%%%%%%%%%%
\frontmatter

%%
% The BIThesis Template for experiment report
%
% Copyright 2020-2022 Yang Yating, BITNP
%
% This work may be distributed and/or modified under the
% conditions of the LaTeX Project Public License, either version 1.3
% of this license or (at your option) any later version.
% The latest version of this license is in
%   http://www.latex-project.org/lppl.txt
% and version 1.3 or later is part of all distributions of LaTeX
% version 2005/12/01 or later.
%
% This work has the LPPL maintenance status `maintained'.
%
% The Current Maintainer of this work is Feng Kaiyu.

\begin{abstract}
  本文......。
  \textcolor{blue}{(摘要是一篇具有独立性和完整性的短文,应概括而扼要地反映出本论文的主要内容。包括研究目的、研究方法、研究结果和结论等,特别要突出研究结果和结论。中文摘要力求语言精炼准确,博士学位论文建议1000~1200字,硕士学位论文摘要建议500~800字。摘要中不可出现参考文献、图、表、化学结构式、非公知公用的符号和术语。英文摘要与中文摘要的内容应完全一致,在语法、用词上应准确无误,语言简练通顺。留学生的英文版博士学位论文中应有不少于3000字的“详细中文摘要”。)}
\end{abstract}

\begin{abstract*}
  In order to exploit.......
\end{abstract*}


%%
% The BIThesis Template for experiment report
%
% Copyright 2020-2022 Yang Yating, BITNP
%
% This work may be distributed and/or modified under the
% conditions of the LaTeX Project Public License, either version 1.3
% of this license or (at your option) any later version.
% The latest version of this license is in
%   http://www.latex-project.org/lppl.txt
% and version 1.3 or later is part of all distributions of LaTeX
% version 2005/12/01 or later.
%
% This work has the LPPL maintenance status `maintained'.
%
% The Current Maintainer of this work is Feng Kaiyu.

\begin{symbols}
  \item[BIT] 北京理工大学的英文缩写
  \item[\LaTeX] 一个很棒的排版系统
  \item[\LaTeXe] 一个很棒的排版系统的最新稳定版
  \item[ctex] 成套的中文\LaTeX{}解决方案,由一帮天才们开发
  \item[$ e^{\pi{}i}+1=0$] 一个集自然界五大常数一体的炫酷方程
\end{symbols}


\MakeTOC

\listoffigures
\listoftables

\mainmatter


%%
% The BIThesis Template for experiment report
%
% Copyright 2020-2022 Yang Yating, BITNP
%
% This work may be distributed and/or modified under the
% conditions of the LaTeX Project Public License, either version 1.3
% of this license or (at your option) any later version.
% The latest version of this license is in
%   http://www.latex-project.org/lppl.txt
% and version 1.3 or later is part of all distributions of LaTeX
% version 2005/12/01 or later.
%
% This work has the LPPL maintenance status `maintained'.
%
% The Current Maintainer of this work is Feng Kaiyu.

\chapter{绪论}

\textcolor{blue}{
  正文包括绪论、论文具体研究内容及结论部分。博士学位论文:一般为6~10万字,其中绪论要求为1万字左右。硕士学位论文:一般为3~5万字,其中绪论要求为0.5万字左右。(外语学科:中文、日文不少于3万字,西文2万字左右。)
}

\textcolor{blue}{
  绪论一般作为第1章。绪论应包括本研究课题的学术背景及其理论与实际意义;本领域的国内外研究进展及成果、存在的不足或有待深入研究的问题;本研究课题的来源及主要研究内容等。
}


\label{chap:intro}
\section{本论文研究的目的和意义}

近年来,随着人们生活水平的不断提高,人们越来越注重周围环境对身体健康的影响。作为服装是人们时时刻刻最贴近的环境,尤其是内衣,对人体健康有很大的影响。由于合时刻刻最贴近的环境,尤其是内衣,对人体健康有很大的影响。由于合成纤维的衣着舒适性、手感性,天然纤维的发展又成为人们关注的一大热点。

……\cite{Takahashi1996Structure,Xia2002Analysis,Jiang1989,Mao2000Motion,Feng1998}

\section{国内外研究现状及发展趋势}
%\label{sec:***} 可标注label

\subsection{形状记忆聚氨酯的形状记忆机理}
%\label{sec:features}

形状记忆聚合物(SMP)是继形状记忆合金后在80年代发展起来的一种新型形状记忆材料\cite{Jiang2005Size}。形状记忆高分子材料在常温范围内具有塑料的性质,即刚性、形状稳定恢复性;同时在一定温度下(所谓记忆温度下)具有橡胶的特性,主要表现为材料的可变形性和形变恢复性。即“记忆初始态-固定变形-恢复起始态”的循环。

固定相只有物理交联结构的聚氨酯称为热塑性SMPU,而有化学交联结构称为热固性SMPU。热塑性和热固性形状记忆聚氨酯的形状记忆原理示意图如图\ref{fig:diagram}所示

\begin{figure}
 \centering
 \includegraphics[width=0.75\textwidth]{figures/figure1}
 \caption{热塑性形状记忆聚氨酯的形状记忆机理示意图}\label{fig:diagram}
\end{figure}


\subsection{形状记忆聚氨酯的研究进展}
%\label{sec:requirements}
首例SMPU是日本Mitsubishi公司开发成功的……。

\subsection{水系聚氨酯及聚氨酯整理剂}

水系聚氨酯的形态对其流动性,成膜性及加工织物的性能有重要影响,一般分为三种类型\cite{Jiang2005Size} ,如表 \ref{tab:category}所示。

\begin{table}
  \centering
  \caption{水系聚氨酯分类} \label{tab:category}
  \begin{tabular*}{0.9\textwidth}{@{\extracolsep{\fill}}cccc}
  \toprule
    类别			&水溶型		&胶体分散型		&乳液型 \\
  \midrule
    状态			&溶解$\sim$胶束	&分散		&白浊 \\
    外观			&水溶型		&胶体分散型		&乳液型 \\
    粒径$/\mu m$	&$<0.001$		&$0.001-0.1$		&$>0.1$ \\
    重均分子量	&$1000\sim 10000$	&数千$\sim 20万$ &$>5000$ \\
  \bottomrule
  \end{tabular*}
\end{table}

\subsubsection{四级节标题}

根据需要,也可设四级节标题

由于它们对纤维织物的浸透性和亲和性不同,因此在纺织品染整加工中的用途也有差别,其中以水溶型和乳液型产品较为常用。另外,水系聚氨酯又有反应性和非反应性之分。虽然它们的共同特点是分子结构中不含异氰酸酯基,但前者是用封闭剂将异氰酸酯基暂时封闭,在纺织品整理时复出。相互交联反应形成三维网状结构而固着在织物表面。
……



\backmatter

\input{./chapters/conclusion.tex}
%%
% The BIThesis Template for experiment report
%
% Copyright 2020-2022 Yang Yating, BITNP
%
% This work may be distributed and/or modified under the
% conditions of the LaTeX Project Public License, either version 1.3
% of this license or (at your option) any later version.
% The latest version of this license is in
%   http://www.latex-project.org/lppl.txt
% and version 1.3 or later is part of all distributions of LaTeX
% version 2005/12/01 or later.
%
% This work has the LPPL maintenance status `maintained'.
%
% The Current Maintainer of this work is Feng Kaiyu.

%
% 如无特殊需要,本页面无需更改


\begin{bibprint}
  \printbibliography[heading=none,notcategory=mypub,resetnumbers=true]
\end{bibprint}


\begin{appendices}
  \section{费马大定理的证明}
  关于此,我确信已发现了一种美妙的证法,可惜这里空白的地方太小,写不下。

  \section{Maxwell Equations}
  因为在柱坐标系下,$\overline{\overline\mu}$是对角的,所以Maxwell方程组中电场$\bf
  E$的旋度

  所以$\bf H$的各个分量可以写为:
  \begin{subequations}
    \begin{eqnarray}
      H_r=\frac{1}{\mathbf{i}\omega\mu_r}\frac{1}{r}\frac{\partial
        E_z}{\partial\theta } \\
      H_\theta=-\frac{1}{\mathbf{i}\omega\mu_\theta}\frac{\partial E_z}{\partial r}
    \end{eqnarray}
  \end{subequations}

  同样地,在柱坐标系下,$\overline{\overline\epsilon}$是对角的,所以Maxwell方程组中磁场$\bf
  H$的旋度
  \begin{subequations}
    \begin{eqnarray}
      &&\nabla\times{\bf H}=-\mathbf{i}\omega{\bf D}\\
      &&\left[\frac{1}{r}\frac{\partial}{\partial
          r}(rH_\theta)-\frac{1}{r}\frac{\partial
          H_r}{\partial\theta}\right]{\hat{\bf
          z}}=-\mathbf{i}\omega{\overline{\overline\epsilon}}{\bf
        E}=-\mathbf{i}\omega\epsilon_zE_z{\hat{\bf z}} \\
      &&\frac{1}{r}\frac{\partial}{\partial
        r}(rH_\theta)-\frac{1}{r}\frac{\partial
        H_r}{\partial\theta}=-\mathbf{i}\omega\epsilon_zE_z
    \end{eqnarray}
  \end{subequations}

  由此我们可以得到关于$E_z$的波函数方程:
  \begin{eqnarray}
    \frac{1}{\mu_\theta\epsilon_z}\frac{1}{r}\frac{\partial}{\partial r}
    \left(r\frac{\partial E_z}{\partial r}\right)+
    \frac{1}{\mu_r\epsilon_z}\frac{1}{r^2}\frac{\partial^2E_z}{\partial\theta^2}
    +\omega^2 E_z=0
  \end{eqnarray}
\end{appendices}

%%
% The BIThesis Template for experiment report
%
% Copyright 2020-2022 Yang Yating, BITNP
%
% This work may be distributed and/or modified under the
% conditions of the LaTeX Project Public License, either version 1.3
% of this license or (at your option) any later version.
% The latest version of this license is in
%   http://www.latex-project.org/lppl.txt
% and version 1.3 or later is part of all distributions of LaTeX
% version 2005/12/01 or later.
%
% This work has the LPPL maintenance status `maintained'.
%
% The Current Maintainer of this work is Feng Kaiyu.

\begin{publications}
  \nocite{myCiteKey}
  \nocite{myCiteKey2}
  \addtocategory{mypub}{myCiteKey}
  \addtocategory{mypub}{myCiteKey2}

  \renewcommand*{\mkbibnamegiven}[1]{%
  \ifitemannotation{highlight}
    {\textbf{#1)}}
    {#1}}

  \renewcommand*{\mkbibnamefamily}[1]{%
    \ifitemannotation{highlight}
      {\textbf{#1}}
      {#1}}

  \printbibliography[heading=none,category=mypub,resetnumbers=true]{}
    % \item\textsc{高凌}. {交联型与线形水性聚氨酯的形状记忆性能比较}[J].
      % 化工进展, 2006, 532-535.(核心期刊)
    
\end{publications}

%%
% The BIThesis Template for experiment report
%
% Copyright 2020-2022 Yang Yating, BITNP
%
% This work may be distributed and/or modified under the
% conditions of the LaTeX Project Public License, either version 1.3
% of this license or (at your option) any later version.
% The latest version of this license is in
%   http://www.latex-project.org/lppl.txt
% and version 1.3 or later is part of all distributions of LaTeX
% version 2005/12/01 or later.
%
% This work has the LPPL maintenance status `maintained'.
%
% The Current Maintainer of this work is Feng Kaiyu.

\begin{acknowledgements}

本论文的工作是在导师……。

\end{acknowledgements}


% 加入目录
\end{document}
